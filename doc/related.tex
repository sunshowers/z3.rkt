\section{Related Work}

\subsubsection{SMT integration.}

Integration with an SMT solver proves to be useful for any programming
language. It enables programmers in that language to use the solver's power to
solve whatever logical constraints arise in a program, without needing to
resort to hand-coding a backtracking algorithm or other cumbersome methods.
The solutions thus obtained can be used in the rest of the program.
Thus it isn't surprising that several such projects exist, most of them
available freely on the Internet. These projects differ mainly in the host
language, the interface, and the constructs they support.

As most languages support some form of interaction with C functions, they can
be said to be already integrated with Z3 (or other SMT solvers) through the C
API. However, we do not consider this to be true integration because it
doesn't actually simplify the job of the programmer and, as noted in
Section~\ref{sec:motiv}, it requires her to deal with the internals of the
solver.

The integration of Scala with Z3~\cite{scalaz3} is one of the most complete
implementations available right now. It provides support for adding new
theories and procedural abstractions, and also takes advantage of Scala's type
system to deal with some type related errors at compile time.  The system has
been used to solve several challenging problems within and outside the group
that developed it. The main disadvantage of this system is that the syntax is
quite different from SMT-LIB, and is sometimes almost as verbose as using the
C bindings.

Z3Py~\cite{z3py} is a new Python interface bundled with Z3: it is much more
pleasant to use than the C version and supports virtually all the features of
Z3.

Symbolic Bit Vectors (SBV)~\cite{sbv} is a Haskell package that can be used to
prove properties about bit-precise Haskell programs. Given a constraint in a
Haskell program, SBV generates SMT-LIB code that can be run against either
Yices or Z3.

Yices-Painless~\cite{yices-painless} integrates Haskell with Yices via its C
API. This project does not support arrays, tuples, lists and user defined data
types yet. Further, the development of the tool seems to be stalled for some
time now (last change to the repository was in January 2011).

The Z3 documentation page lists bindings to other languages like OCaml. These
bindings correspond almost one-to-one with the C API, and thus they suffer
from the same disadvantages.
