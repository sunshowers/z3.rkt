\section{Related Work}
Integration with a SMT solver is desirable for any
programming language. It enables the programmers in that
language to use integrated SMT solver's power to solve
logical constraints arising in a program.  The solutions thus
obtained can be used in the rest of the program. The
alternative would be to hand-code a solution using
backtracking or other methods, which is time-consuming and
inefficient for complex constraints. Thus, it is not
surprising that several such projects exist, most of them
available freely on the Internet. These projects differ
mainly in the host language, the interface, and the
constructs they support.

As most languages support some
form of interaction with C functions, theoretically these
languages are already integrated with Z3 (or other SMT
solvers) through the C API. However, we do not consider
this to be true integration because it doesn't actually simplify the job
of the programmer and, as noted in Section~\ref{sec:motiv}, it
requires her to deal with the internals of the solver.

The integration of Scala with Z3~\cite{scalaz3} is one of the
most complete implementations available as of now. It
provides support for adding new theories and procedural
abstractions, and also takes advantage of Scala's type system
to deal with type related errors at compile time.  The system
has been used to solve several challenging problems within
and outside the group that developed it. The main
disadvantage of this system is that the syntax is quite
different from SMT-LIB, and is almost as verbose as using the
C bindings.

Symbolic Bit Vectors (SBV)~\cite{sbv} is a Haskell package
that can be used to prove properties about bit-precise
Haskell programs. Given a constraint in a Haskell program,
SBV generates SMT-LIB code that can be run against either
Yices or Z3.  However, it only works for bitvector formulas.

Yices-Painless~\cite{yices-painless} integrates Haskell with
Yices via its C API. This project does not support
arrays, tuples, lists and user defined data types yet. Further,
the development of the tool seems to be stalled for some time
now (last change to the repository was in January 2011).

The Z3 documentation page lists bindings
to other languages like OCaml and Python. These bindings correspond
almost one-to-one with the C API, and thus they suffer from the same
disadvantages.
